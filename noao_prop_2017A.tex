% NOAOPROP.TEX -- template for NOAO STANDARD telescope proposals.
% Revised for the 2017A observing semester (Feb - Jul 2017)
% For later semesters, the current form may be obtained from our Web
% page at https://na01.safelinks.protection.outlook.com/?url=http%3A%2F%2Fwww.noao.edu%2Fnoaoprop%2Fnoaoprop.html&data=01%7C01%7Ckap146%40pitt.edu%7C6f8fc2f751be47ae990208d3e317779d%7C9ef9f489e0a04eeb87cc3a526112fd0d%7C1&sdata=u2D6xsWjfC3h8uqUnBE2tcWoSlasAb52wEepSjj84KU%3D&reserved=0

% DEADLINE FOR SUBMISSIONS: 11:59pm MST FRIDAY SEPTEMBER 30, 2016

% This form can NOT be used to submit Gemini proposals.  Please
% see:   https://na01.safelinks.protection.outlook.com/?url=http%3A%2F%2Fwww.noao.edu%2Fnoaoprop%2Fhelp%2Fpit.html&data=01%7C01%7Ckap146%40pitt.edu%7C6f8fc2f751be47ae990208d3e317779d%7C9ef9f489e0a04eeb87cc3a526112fd0d%7C1&sdata=T5Kn7s51S%2BhhoQUPvn8AouM82nG913859XgcfybqItQ%3D&reserved=0

% This LaTeX template has been returned to you upon request through
% our Web-based proposal form.  The template has been customized for
% you based on the run information that you entered via the Web form.
% As an option you may complete this form locally and submit it by
% email following the instructions below.  If the run information is
% incomplete you should go back to the Web form, complete all the
% suggested run information and then obtain another customized 
% template before proceeding. 
%
% WEB FORM SUBMISSION:
% If you have not modified this document locally, we encourage web
% submissions using the "Proposal Submission" button on the web
% proposal form.  Be sure to click on the second prompt on the
% following page to complete submission.   Skip to step #3 below
% for some additional information.
%
% FILE UPLOAD SUBMISSION:
% If you have modified this document locally and are ready to submit, 
% you do not need to return to the web form.  Instead, you may submit
% this template and figure files by following the instructions 
% below.
%
%   1. Before submitting this form electronically, run it through latex
%      and print it out to make certain that it looks the way that you
%      wish the review panel to see it.
%   
%   2. Upload THIS file (NOT the PostScript or PDF version) and
%      all figures on the web at https://na01.safelinks.protection.outlook.com/?url=http%3A%2F%2Fwww.noao.edu%2Fnoaoprop%2Fsubmit%2F&data=01%7C01%7Ckap146%40pitt.edu%7C6f8fc2f751be47ae990208d3e317779d%7C9ef9f489e0a04eeb87cc3a526112fd0d%7C1&sdata=MLKb%2FXAabunU%2FZANzvkzworL8Bcb6emrdrP1bHi72sU%3D&reserved=0
%
%   3. If the proposal is a thesis or if the principal investigator is a
%      a graduate student, the student's faculty advisor must complete
%      the online form at https://na01.safelinks.protection.outlook.com/?url=http%3A%2F%2Fwww.noao.edu%2Fnoaoprop%2Fthesis%2F&data=01%7C01%7Ckap146%40pitt.edu%7C6f8fc2f751be47ae990208d3e317779d%7C9ef9f489e0a04eeb87cc3a526112fd0d%7C1&sdata=qsqNcPSGwoOImFPO9wBKLbKmHbY7P%2FeYeaqOpjMl0cU%3D&reserved=0
%      This form should be completed within two business days after the
%      deadline.  Graduate students proposing thesis observations should
%      consult NOAO policies concerning thesis programs and travel support.
%
%   4. QUESTIONS?  If you have questions about submitting your proposal
%      send email to noaoprop-help@noao.edu.  Information about
%      instrumentation can be found at the NOAO Web 
%      site at https://na01.safelinks.protection.outlook.com/?url=http%3A%2F%2Fwww.noao.edu%2Fnoaoprop%2Fnoaoprop.html&data=01%7C01%7Ckap146%40pitt.edu%7C6f8fc2f751be47ae990208d3e317779d%7C9ef9f489e0a04eeb87cc3a526112fd0d%7C1&sdata=u2D6xsWjfC3h8uqUnBE2tcWoSlasAb52wEepSjj84KU%3D&reserved=0.  Specific 
%      instrumentation or facility questions for CTIO, KPNO, and 
%      Keck, can also be sent to the respective sites at ctio@noao.edu, 
%      kpno@noao.edu, and keck@noao.edu.
%      Gemini questions can be  emailed to the US mirror scientists 
%      at nssc@noao.edu.
%
%   5. When your proposal is received at NOAO you will be sent an
%      automatic email message verifying its receipt along with a
%      proposal ID number.  If you do not receive this message within
%      15 minutes of the time you sent your proposal send email to
%      noaoprop-help@noao.edu for assistance.  You may track your
%      proposal processing by the proposal ID number at the following
%      web page:   https://na01.safelinks.protection.outlook.com/?url=http%3A%2F%2Fwww.noao.edu%2Fcgi-bin%2Fnoaoprop%2Fpropstatus&data=01%7C01%7Ckap146%40pitt.edu%7C6f8fc2f751be47ae990208d3e317779d%7C9ef9f489e0a04eeb87cc3a526112fd0d%7C1&sdata=2eX60lnxRn2nFQGC2AM3zauvMtQO%2FXY0GaFnjQBX37Y%3D&reserved=0
% ___________________________________________________________________
% THE FORM STARTS HERE
%

% Please do not modify or delete this line.
\documentclass[11pt]{article}
\usepackage{nprop30}


% Please do not modify or delete this line.
\begin{document}

% Uncomment the following line and enter a previous semester and ID
% (e.g. 01B-0987) if you wish to flag this proposal as a resubmission
\pastid{12B-0500}

% Please do not modify or delete this line.
\proposaltype{Standard}
% Do not delete the following uncommented line which is required for Gemini 
% proposals.  This line indicates whether or not the proposal is being 
% submitted to multiple partner countries - only "yes" or "no" is valid.
\geminidata{mpartners}{no}

%%%%%%%%%%%%%%%%%%%%%%%%%%%%%%%%%%%%%%%%%%%%%%%%%%%%%%%%%%%%%%%%%%%%%

% SCIENTIFIC CATEGORIES
%
% Please select a "scientific category" that best describes your
% program by uncommenting only ONE of the selections below.  Your
% \sciencecategory selection will be used to assign a review panel to
% your proposal.  DO NOT MODIFY THE SELECTION YOU UNCOMMENT.  A
% description of each of these categories is available on our Web page
% at https://na01.safelinks.protection.outlook.com/?url=http%3A%2F%2Fwww.noao.edu%2Fnoaoprop%2Fhelp%2Fscicat.html&data=01%7C01%7Ckap146%40pitt.edu%7C6f8fc2f751be47ae990208d3e317779d%7C9ef9f489e0a04eeb87cc3a526112fd0d%7C1&sdata=JjpbJT%2FIsSrlnASdvE4XOaFYz2x3LoagV13ydy3v0wo%3D&reserved=0

% EXTRA-GALACTIC LIST (do not uncomment this line)
%\sciencecategory{Active Galaxies}
\sciencecategory{Cosmology}
%\sciencecategory{Large Scale Struc.}
%\sciencecategory{Clusters of Galaxies}
%\sciencecategory{High Z Galaxies}
%\sciencecategory{Low Z Galaxies}
%\sciencecategory{Resolved Galaxies}
%\sciencecategory{Stellar Pops (EGAL)}
%\sciencecategory{EGAL - Other}

% GALACTIC/LOCAL GROUP LIST (do not uncomment this line)
%\sciencecategory{Star Clusters}
%\sciencecategory{Stellar Pops (GAL)}
%\sciencecategory{HII Reg., PN, etc.}
%\sciencecategory{ISM}
%\sciencecategory{Star Forming Regions}
%\sciencecategory{Young Stellar Obj.}
%\sciencecategory{Massive Stars}
%\sciencecategory{Low Mass Stars}
%\sciencecategory{Stellar Remnants}
%\sciencecategory{Galactic - Other}

% SOLAR SYSTEM LIST (do not uncomment this line)
%\sciencecategory{Kuiper Belt Objects}
%\sciencecategory{Small Bodies \& Moons}
%\sciencecategory{Planets}
%\sciencecategory{Extrasolar Planets}
%\sciencecategory{Solar System - Other}

%%%%%%%%%%%%%%%%%%%%%%%%%%%%%%%%%%%%%%%%%%%%%%%%%%%%%%%%%%%%%%%%%%%%%%

% TITLE
%
% Give a descriptive title for the proposal in the \title command.
%
% Note that a title can be quite long; LaTeX will break the title into
% separate lines automatically.  If you wish to indicate line breaks
% yourself, do so with a `\\' command at the appropriate point in
% the title text.  Use both upper and lower case letters (NOT ALL CAPS).

%\title{Final Two Host Galaxy Observations for ``SweetSpot"}
\title{Final Calibrations for the ``SweetSpot" Survey}


%%%%%%%%%%%%%%%%%%%%%%%%%%%%%%%%%%%%%%%%%%%%%%%%%%%%%%%%%%%%%%%%%%%%%%

% ABSTRACT
%
% Give a general abstract of the scientific justification appropriate
% for a non-specialist.  Write between the \begin{abstract} and 
% \end{abstract} lines.  Limit yourself to approximately 175 words.
% Abstracts of accepted proposals will be made publicly available.

% DO NOT remove the \begin{abstract} and \end{abstract} lines.

\begin{abstract}

The SweetSpot program (2012B-0500) observed 114 SNeIa during its six semester run. 
Host galaxy template images were gathered during the main survey and in additional time allotted in 2015B and 2016A semesters (2012B-0500, 2015B-0347). 
Only two templates are left until we have collected all necessary reference images to allow accurate measurement of the SN flux and thus distance for the supernova observed in the last semesters of SweetSpot.
With the addition of standard stars of varying colors, we will also improve the calibration of WIYN's PWV system. 

\end{abstract}

%%%%%%%%%%%%%%%%%%%%%%%%%%%%%%%%%%%%%%%%%%%%%%%%%%%%%%%%%%%%%%%%%%%%%%

% SUMMARY OF OBSERVING RUNS REQUESTED
%
% List a summary of the details of the observing runs being requested,
% for UP TO SIX runs.  The parameters for each run are segregated
% between \begin{obsrun} and \end{obsrun} lines.  Please be sure
% that the information is isolated properly for each run.
%
%   \begin{obsrun}
%   \telescope{}        % For example, \telescope{KP-4m}
%   \instrument{}       % For example, \instrument{ECHUV + T2KB}
%   \numnights{}        % For example, \numnights{6}
%   \lunardays{}        % For example, \lunardays{grey}
%   \optimaldates{}     % For example, \optimaldates{Sep - Nov}
%   \acceptabledates{}  % For example, \acceptabledates{Aug - Jan}
%   \end{obsrun}
%
% The following telescope identifiers MUST be used in the \telescope{}
% field.  Some of the telescope identifiers must include an observatory
% code as well.
%
% CTIO: CT-4m, SOAR, CT-1.5m, CT-1.3m, CT-0.9m
% KPNO: KP-4m, WIYN, KP-0.9m
% AAO: AAT
% CHARA: CHARA
%
% Select the instrument and detector identifiers from the list on our
% Web page at https://na01.safelinks.protection.outlook.com/?url=http%3A%2F%2Fwww.noao.edu%2Fnoaoprop%2Fhelp%2Ffacilities.html&data=01%7C01%7Ckap146%40pitt.edu%7C6f8fc2f751be47ae990208d3e317779d%7C9ef9f489e0a04eeb87cc3a526112fd0d%7C1&sdata=xzUfvUtUpc%2BsBGd9VEUtvVAoVY5SzAPxhuL7PpoDJf4%3D&reserved=0.
% The correct codes MUST be used to ensure your correct
% instrument + detector combination.
%
% \numnights should give the number of nights of the run (for queue
% observations, use fractional 10-hour equivalent nights, e.g.
% 3 hours = 0.3 nights).  Formats such as 5x0.5 are acceptable.
%
% \lunardays should contain the word "darkest", "dark", "grey", or
% "bright", which in turn reflects the number of nights from new moon
% where darkest<=3, dark<=7, grey<=10, bright<=14.  Particular lunar 
% phase requirements dictated by the science program (e.g., "<=12", 
% "+9, -6", or "full moon more than 2 hours away from Taurus") should
% be noted in the "scheduling constraints or non-usable dates" section 
% below.
%
% \optimaldates should contain the range of OPTIMAL months, as shown 
% below.
%
% \acceptabledates should give the range of ACCEPTABLE months (i.e.,
% you would not accept time outside those limits).
% NOTE THAT DUE TO INSTRUMENT BLOCKING RESTRICTIONS YOU SHOULD MAKE 
% THIS RANGE AS GENEROUS AS POSSIBLE.
%
% For QUEUE-SCHEDULED observations, you may set the date range to the 
% full semester range and set \lunardays to the brightest moon your
% observations could tolerate if the program were scheduled classically.
%
% To enter the acceptable and optimal date ranges, please use two
% dash-separated months with 3-letter abbreviations for the month
% (Jan, Feb, Mar, Apr, May, Jun, Jul, Aug, Sep, Oct, Nov, Dec).
% For example:  \optimaldates{Nov - Dec}.
% We appreciate your help in not using vague range specifications
% like "October dark run" or "mid-January" which will require human
% intervention.
%
% FOR LONGTERM STATUS PROPOSALS SPECIFY ONLY THE RUNS FOR THE CURRENT
% SEMESTER, AND NOT FOR ANY SUBSEQUENT SEMESTERS.

% DO NOT remove any of the \begin{obsrun} and \end{obsrun} blocks, 
% even if the blocks are empty.

\begin{obsrun}
\telescope{WIYN}
\instrument{WHIRC}
\numnights{1}
\lunardays{bright}
\optimaldates{April - May}
\acceptabledates{Mar - May}
\end{obsrun}

\begin{obsrun}
\telescope{}
\instrument{}
\numnights{}
\lunardays{}
\optimaldates{}
\acceptabledates{}
\end{obsrun}

\begin{obsrun}
\telescope{}
\instrument{}
\numnights{}
\lunardays{}
\optimaldates{}
\acceptabledates{}
\end{obsrun}

\begin{obsrun}
\telescope{}
\instrument{}
\numnights{}
\lunardays{}
\optimaldates{}
\acceptabledates{}
\end{obsrun}

\begin{obsrun}
\telescope{}
\instrument{}
\numnights{}
\lunardays{}
\optimaldates{}
\acceptabledates{}
\end{obsrun}

\begin{obsrun}
\telescope{}
\instrument{}
\numnights{}
\lunardays{}
\optimaldates{}
\acceptabledates{}
\end{obsrun}


% If there are scheduling constraints or non-usable dates for any of
% the runs specified, (i.e., other than the default lunar phase 
% requirements or when your object is up) please give the dates by 
% filling in the curly braces in \unusabledates{}.  Note here if you 
% are requesting runs in an "either/or" situation, e.g. run 1 or run 2, 
% but not both. This is also the place to advise us of any special 
% constraints which affect the scheduling of your observing run (e.g. 
% "schedule run #1 before run #2" or "run dates must be coordinated 
% with HST observations").
% 
% Please limit your text to six lines on the printed copy.

\unusabledates{
The two targets are at very different locations on the sky but can be observed on one well placed night in April or May. 
However, they may also be done on 2 separate partial nights, including T$\&$E nights.}

%%%%%%%%%%%%%%%%%%%%%%%%%%%%%%%%%%%%%%%%%%%%%%%%%%%%%%%%%%%%%%%%%%%%%%%

% INVESTIGATOR'S (PI AND CoI) INFORMATION BLOCKS
%
% Please give the PI's name (first name first followed by middle
% initial and last name), affiliation, department and complete mailing
% address, as well as an email address.  Also give a complete phone
% number, and a number for a fax machine if you have access to one.
% You must also indicate the principal investigator's status with one
% of the one-letter codes inside the \invstatus{} curly braces, as
% indicated below.
%
% The affil{}, \department{}, \address{} (use a comma separate list as
% needed), \city{}, \state{}, \zipcode{}, and \country{} (for non-US
% addresses) fields will be used together as your full postal mailing
% address.  Please be sure this information is complete.  Note that
% some institutions will not deliver postal mail if a department is
% not included in the postal mailing address.  Non-US addresses should
% include the country and any local postal codes.
%
% The fax number does not print on the form.
%
% For each CoI please include a name, affiliation, email address and
% investigator's status within the \begin{CoI} and \end{CoI} lines.
%
% For each \invstatus{} field, please fill in the appropriate
% investigator status code from the following list.  If the investigator
% is a graduate student, indicate "T" if THIS proposal is related to a
% thesis project, or "G" otherwise.  This code should represent the
% status of the individuals at the time of the proposal submission.
% This information is necessary to assist us with our required reporting
% to the NSF.
%
% \invstatus{P} % investigator has obtained PhD or equivalent
% \invstatus{T} % investigator is grad student, proposal is thesis
% \invstatus{G} % investigator is grad student, proposal not thesis
% \invstatus{U} % investigator is an undergraduate student
% \invstatus{O} % investigator has other status (none of the above)
%
% DO NOT remove the \begin{PI} and \end{PI}.  Only one individual's
% name per \name field is allowed.
%
% Investigator names will now appear on page 2 of the printed proposal.
% Do not remove this line.
\investigators

\begin{PI}
\name{Michael Wood-Vasey}
\affil{University of Pittsburgh}
\department{Physics and Astronomy}
\address{3941 O'Hara S}
\city{Pittsburgh}
\state{PA}
\zipcode{15260}
\country{USA}
\email{wmwv@pitt.edu}
\phone{4126249000}
\fax{}
\invstatus{P}
\end{PI}

\begin{CoI}
\name{Kara Ponder}
\affil{University of Pittsburgh}
\email{kap146@pitt.edu}
\invstatus{T}
\end{CoI}

\begin{CoI}
\name{Tom Matheson}
\affil{NOAO}
\email{matheson@noao.edu}
\invstatus{P}
\end{CoI}

\begin{CoI}
\name{Dick Joyce}
\affil{NOAO}
\email{joyce@noao.edu}
\invstatus{P}
\end{CoI}

\begin{CoI}
\name{Saurabh Jha}
\affil{Rutgers University}
\email{saurabh@physics.rutgers.edu}
\invstatus{P}
\end{CoI}

\begin{CoI}
\name{Peter Garnavich}
\affil{University of Notre Dame}
\email{pgarnavi@nd.edu}
\invstatus{P}
\end{CoI}

\begin{CoI}
\name{Lori Allen}
\affil{NOAO}
\email{lallen@noao.edu}
\invstatus{P}
\end{CoI}


%%%%%%%%%%%%%%%%%%%%%%%%%%%%%%%%%%%%%%%%%%%%%%%%%%%%%%%%%%%%%%%%%%%%%%

% In the following "essay question" sections, the delimiting pieces of
% markup (\justification, \expdesign, etc.) act as LaTeX \section*{}
% commands.  If the author wanted to have numbered subsections within
% any of these, LaTeX's \subsection could be used.
%
% DO NOT REDUCE THE FONT SIZE, and do not otherwise fiddle with the
% format to get more on a page.  We will reset any changes back to the
% default font.

% SCIENTIFIC JUSTIFICATION
%
% Give the scientific justification for the proposed observations.
% This section should consist of paragraphs of text followed by any
% references and up to three figures and captions.  Be sure to include
% overall significance to astronomy.  THE SCIENTIFIC JUSTIFICATION
% SHOULD BE LIMITED TO ONE PAGE (the review panels have requested that
% we not send them more than one page), with up to two additional pages
% for references, figures (no more than three), and captions. 

% If you wish to use our "reference" environment, follow the following
% example (journal commands are compatible with AASTeX v4.0):
%
%\begin{references}
%\reference Armandroff \& Massey 1991 \aj, 102, 927.
%\reference Berkhuijsen \& Humphreys 1989 \aap, 214, 68.
%\reference Massey 1993 in Massive Stars: Their Lives in the 
% Interstellar Medium (Review), ed. J. P. Cassinelli and E. B. 
% Churchwell, p. 168.
%\reference Massey \& Armandroff 1999, in prep.
%\end{references}

% Only EPS figures may be included with your proposal.  In order to
% include an EPS plot, you should use the LaTeX "figure" environment.
% The plot file is included with the \plotone{FILENAME} command; two
% side-by-side plot files can be included by typing
% \plottwo{FILENAME1}{FILENAME2}.  Use \caption{} to specify a caption.
% The \epsscale{} command can be used to scale \plotone plots if they
% appear too large on the printed page.  Contact us if you have any
% figure questions or encounter any problems with figures
% (noaoprop-help@noao.edu).
%
% \begin{figure}
% \epsscale{0.85}
% \plotone{sample.eps}
% \caption{Sample figure showing important results.}
% \end{figure}
%
% If you need to rotate or make other transformations to a figure, you
% may use the \plotfiddle command:
% \plotfiddle{PSFILE}{VSIZE}{ROTANG}{HSCALE}{VSCALE}{HTRANS}{VTRANS}
% \plotfiddle{sample.eps}{2.6in}{-90.}{32.}{32.}{-250}{225}
% where HSCALE and VSCALE are percentages and HTRANS and VTRANS are
% in PostScript units, 72 PS units = 1 inch.
%
% Note that the Web form provides several useful and simple figure 
% options.

\sciencejustification
{\bf How Standard are Type Ia Supernovae Standard Candles in the Near-Infrared?} 
In our quest toward precision measurements of dark energy with Type Ia supernovae (SNeIa)~\cite{astier06,wood-vasey07,kowalski08,hicken09b,kessler09,conley11,betoule14} we have became limited by 
systematic errors due to our incomplete understanding of SNIa colors, dust, and 
their host environments.
SNeIa are superior distance
indicators in the NIR, with more standard peak $H$ magnitudes and
relative insensitivity to reddening
\cite{meikle00,krisciunas04a,krisciunas07}. 
As a result, unlike optical Type Ia SNe, which are {\em standardizable} candles,
 NIR SNe Ia
appear to be truly {\em standard} candles at the $\sim0.10$--$0.15$ mag
level ($\sim5$--$7\%$ in distance)
\cite{krisciunas04a,krisciunas05a,wood-vasey08,folatelli10,weyant11,barone-nugent12,kattner12}.

Motivated by this opportunity, the ``SweetSpot'' (PI Wood-Vasey; \cite{weyant11}) and the ``CSP-II'' (PI M. Phillips; \cite{contreras10,stritzinger11}) programs were undertaken to build a comprehensive sample of SNeIa observed in the NIR in the nearby Hubble flow.
The scientific goals of SweetSpot are:
{\bf (1)}
Testing if SNeIa are better standard candles in the NIR.
{\bf (2)}
Breaking the color-dust degeneracy with NIR observations.
{\bf (3)}
Investigating the nature of SNIa host galaxies using NIR and optical observations
{\bf (4)}
Connecting local flows and motions with SNIa NIR to galaxies and convergence.

Throughout the SweetSpot program, we have been obtaining host galaxies templates for SNeIa from the previous year with significant underlying host galaxy flux.   
We here request time an additional semester to observe the host galaxies from the final year of SweetSpot (2014B--2015A) that we were unable to gather in the previous two semesters.  
Without the final host galaxy template references, 
one SN at our target redshift of 0.05 will only have an H-band light curve, which will make it difficult or impossible to calibrate and thus unusable. 
The K$_s$ observations are used to help constrain dust and color, but only a small percentage of our targets have K$_s$ observations.  
It is important to take advantage of the ones that do such as the second target, which needs only a K$_s$ template. 
Obtaining final host galaxy images free from the contamination of the SN light will be critical in obtaining accurate apparent brightness measurements for the SN light curve, and will also help provide measurements of the host galaxy stellar mass along with detailed morphology.

%We only need a single filter in each field, so we do have some usable data of the supernova, but we will not be able to examine colors. 
%One target is at redshift of 0.05 and only have JH observations. So we will not be able to calibrate our flux well with no color information. 

%ASASSN-15fs possibly has pre-B Max observations in Ks. (only half of the LC point have Ks)



%{\bf How do SNIa NIR Luminosities Correlate with Local Host Environment?}\\
%During the course of the SweetSpot program, new results have highlighted the importance of obtaining very local host galaxy spectroscopic measurements.  \cite{rigault13} found that measurements of the local H-$\alpha$ surface density (within 1~kpc) help separate SNeIa into two populations in their optical residual brightness after correction for the optical lightcurve width-luminosity relationship.  However, the global spectroscopic properties of the host galaxies were not sufficient; it was the local star formation rate indicators, specifically H-$\alpha$, that was necessary to separate the SNIa populations and obtain more precise and accurate distance indicators.
%
%It its thus most timely to test this relationship in the NIR.  We will obtain optical HexPak spectra of the SN host galaxies we are observing for final imaging templates to provide spatial-resolved H-$\alpha$, O[II]/O[III], D4000\AA, and H-$\beta$ measurements to study key indicators of recent star formation and metallicity.  Figure~1 shows an example of the HexPak coverage of SN~2014aa.  
%
%Obtaining these HexPak spectra in conjunction with the WHIRC reference images will yield significant operational efficiencies in target acquisition and calibration.

{\bf Improved Calibration:}\\
A dual-band GPS system was installed at KPNO in 2015 March to measure the variable amount of precipitable water vapor (PWV) in the atmosphere.  
This system is a key part of determining the full effective system transmission function in the NIR (and in general for $\lambda>700$~nm), where the edges of the filter bandpasses can be affected by the variable absorption lines from water vapor.  
Observing the same fields for these 2 supernovae and the standard SweetSpot calibration star fields with the dual-band GPS system running for another semester will provide important information to tie down our WIYN+WHIRC magnitudes of the stars in the SN fields we use for calibration.  
Furthermore, we plan to use additional standard stars with a wider range of colors from 0--1 mags at airmasses of 1--2. 
These will provide improved calibration since all previous standard stars have approximately the same color. 
We will use the focus of spending this additional time with the system to provide to all KPNO the water absorption spectral response function based on the observations of the GPS PWV system.



\clearpage
\begin{thebibliography}{11}
\expandafter\ifx\csname natexlab\endcsname\relax\def\natexlab#1{#1}\fi

\bibitem{astier06}
{Astier}, P., et al. (2006) {\it A\&A}, 447, 31.

\bibitem{barone-nugent12}
{Barone-Nugent}, R.~L. et al. (2012) {\it MNRAS}, 425, 1007.

\bibitem{betoule14}
{Betoule}, M. et al. (2014) {\it A\&A}, 568, 22.

%\bibitem{calzetti94}
%{Calzetti}, D. (1994), {\it ApJ}, 429, 582.
%
%\bibitem{calzetti13}
%{Calzetti}, D. (2013) {in \it Secular Evolution of Galaxies}, arXiv:1208.2997

\bibitem{conley11}
{Conley}, A., et al. (2011), ApJS, 192, 1

\bibitem{contreras10}
{Contreras}, C., et al. (2010) {\it AJ}, 139, 519.

\bibitem{folatelli10}
{Folatelli}, G., et al. (2010) {\it AJ}, 139, 120.

\bibitem{hicken09b}
{Hicken}, M., et al. (2009) {\it ApJ}, 700, 1097.

\bibitem{kattner12}
{Kattner}, S. et al. (2012) {\it PASP}, 124, 114.

\bibitem{kauffmann03}
{Kauffmann}, G., {Heckman}, T.~M., and White, S.~D.~M. (2003) {\it MNRAS}, 341, 33.

\bibitem{kelly10}
{Kelly}, P. L., et al. (2010) {\it ApJ}, 715, 743.

\bibitem{kelly15}
{Kelly}, P. L., et al. (2015) {\it Science}, 2015 Mar 27. arXiv:1410.0961.

\bibitem{kennicut98}
{Kennicutt}, R.~C. (1998) {\it ARA\&AA}, 36, 189.

\bibitem{kessler09}
{Kessler}, R., et al. (2009) {\it ApJS}, 185, 32.

\bibitem{kowalski08}
{Kowalski}, M., et al. (2008) {\it ApJ}, 686, 749.

\bibitem{krisciunas07}
{Krisciunas}, K., et al. (2007) {\it ApJ}, 133, 58.

\bibitem{krisciunas04a}
{Krisciunas}, K., {Phillips}, M.~M., and {Suntzeff}, N.~B. (2004) {\it ApJL}, 602, L81.

\bibitem{krisciunas05a}
{Krisciunas}, K., et al. (2005) {\it AJ}, 130, 350.

\bibitem{meikle00}
{Meikle}, W.~P.~S. (2000), {\it MNRAS}, 314, 782.

%\bibitem{poggianti97}
%{Poggianti}, B. (1997), {\it A\&A}, 325, 1025.

%\bibitem{rigault13}
%{Rigault}, M. (2013), {\it A\&A}, 560, 66.

%\bibitem{rigault15}
%{Rigault}, M. (2015), {\it ApJ}, 802, 20.

\bibitem{stritzinger11}
{Stritzinger}, M., et al. (2011), {\it AJ}, 142, 156.

%\bibitem{sullivan10}
%{Sullivan}, M. (2010), {\it MNRAS}, 406, 782.

\bibitem{weyant11}
Weyant, A. et al. (2011) {\it ApJ}, 784, 105.

\bibitem{wood-vasey07}
{Wood-Vasey}, W.~M., et al. (2007) {\it ApJ}, 666, 694

\bibitem{wood-vasey08}
Wood-Vasey, W.~M., et al. (2008) {\it ApJ}, 689, 377.



\end{thebibliography}

\clearpage


% EXPERIMENTAL DESIGN
%
% This section should consist of text only (no figures).
% There is a limit of one page of printed text.

% Describe your overall observational program.  How will these 
% observations contribute toward the accomplishment of the goals 
% outlined in the science justification?  If you've requested 
% long-term status, justify why this is necessary for successful 
% completion of the science.
%
% NOTE: In previous versions of the proposal form, this section
% requested details about the use of non-NOAO observing facilities. 
% Such information should now be entered in the following "Other
% Facilities" section.

\expdesign

We request a total of 1 night for WHIRC observations of host galaxies of SNeIa and standard stars.  

We will observe the 2 host galaxies from SweetSpot that will
require template observations due to significant inferred host galaxy light contributions at the location of the SNeIa.
We are only requiring one filter per object. 
We have gathered the other 1--2 filters in previous semesters but were unable to observe these two due to unforeseen time constraints. 

We will also observe $\sim$ 3 standard stars in all 3 filters throughout the night.

These observations will be undertaken with WHIRC in $J$, $H$, and $K_s$
and provide reference flux values critical to obtaining
SN lightcurves and improved PWV calibration.  
The host galaxy observations will also generate
maps of the host galaxy at the high resolutions offered by the WIYN+WHIRC system for general study of detailed NIR morphology and host galaxy mass.

% PROPRIETARY PERIOD
% 
% Enter the proprietary period for your data between the braces.
% The normal duration is 18 months from when the data are taken at
% the telescope.  Requests for longer proprietary periods must
% be approved by the NOAO Director.

\proprietaryperiod{18 months}


% OTHER FACILITIES OR RESOURCES
% 
% This section should consist of text only (no figures).
% Please limit to about a half page of printed text.
% 
% 1) We are interested in understanding how observations made through
% NOAO observing opportunities complement or support data from other
% facilities both on the ground and in space.   We will use this
% information to guide the evolution of the NOAO program; it will not
% affect the success of your proposal in the evaluation process.
% Please describe how the proposed observations complement data from  
% other facilities, including private observatories and both ground-
% and space-based telescopes.  In addressing this question, take a
% broad view of your research program.  Are the data to be obtained 
% through this proposal going to help select samples for detailed
% observations using larger telescopes or from space observatories?
% Are these data going to be directly combined with data obtained
% elsewhere to test a hypothesis?  Will these observations have
% relevance to other observations, even though the proposal stands
% on its own?  For each of these other facilities, indicate the nature
% of the observations (yours or those of others), and describe the
% importance of the observations proposed here in the context of the
% entire program.
%
% 2) Do you currently have a grant that would provide resources
% to support the data processing, analysis, and publication of the
% observations proposed here?

\otherfacilities

1) These observations will complement the data from other nearby supernova groups such as KAIT, the CfA Supernova group, and the Carnegie Supernova Project to produce the most complementary data sets to enable explorations of optical vs. NIR distance estimation, color, and host galaxy properties.
The first steps toward higher redshift are currently being undertaken on {\it HST} through the RAISIN project (PI R. Kirshner).
Farther in the future and going farther in the past, the nearby NIR SNIa set will provide a reference anchor for future higher-redshift restframe NIR work with {\it JWST} and {\it WFIRST}. ANY PWV WORK THAT CAN BE CITED?

2) The PI is currently funded by NSF AST-1028162 to carry out the SweetSpot program and related nearby SNIa work.  This grant will continue to support graduate students to do the observations, analysis and publications for these proposed observations.  

% LONG-TERM DETAILS
%
% If you are requesting long-term status for this proposal briefly 
% state the requirements for telescope time (telescope, instrument, 
% number of nights) needed in subsequent semesters to complete this 
% project in \longtermdetails (be sure to uncomment the 
% \longtermdetails line below).
%
% If this is a long-term request you MAY ALSO NEED to modify the
% \proposaltype keyword at the top of this form changing "Standard"
% to "Longterm" (where is says "Please do not modify or delete this 
% line!").  It is this keyword that will flag this proposal as a 
% long-term status request, regardless of what may be entered here 
% in \longtermdetails!
%
% If this is not a long-term status request then please ignore this
% section.
%
%
%\longtermdetails


% PAST USE
%
% How effectively have you used the facilities available through NOAO
% in the past?
% List allocations of telescope time on facilities available through 
% NOAO to the Principal Investigator during the past 2 years, together 
% with the current status of the data (cite publications where 
% appropriate).  Mark any allocations of time related to the current 
% proposal with a \relatedwork{} command.

\thepast

This proposal will provide the final data necessary to complete our NOAO Survey 
program ``Type Ia Supernovae in the Near-Infrared: A Three-Year Survey toward a One Percent Distance Measurement with WIYN+WHIRC'' \relatedwork{2012B-0500},
and a final installment of WHIRC host galaxy templates that went uncompleted in the follow up program
``Final Host Galaxy Observations for `SweetSpot': Calibrating the Supernova Host Galaxy Light and Environment''
\relatedwork{2015B-0347}.
We observed 114 SNeIa and 1 SN Ibn (SN 2015G) during the main program.
This first results from this Survey were published in \cite{weyant11}.  
A first data release paper with 74 SNeIa and 31 light curves is currently in preparation and anticipated to be submitted for publication this fall.
Final data reductions of all targets have already begun in preparation for the second and full data release

PI Wood-Vasey was involved in the 6-year ESSENCE Supernova Survey (PI Suntzeff) that used the CTIO 4.0-m Blanco telescope to discover and study 213 Type Ia Supernovae to measure the dark energy equation-of-state during the past 8 billion years.  This survey has so far led to 10 refereed publications that have been cited a combined 1,657 times.

% 0 + 29 + 47 + 19 + 6 + 476 + 746 + 224 + 68 + 42
G. Narayan et al. (2016), ApJS, 224, 36.  \\  % 0
R. J. Foley et al. (2009), AJ, 137, pp. 3731-3742.  \\  % 29
R. J. Foley et al. (2008), ApJ, 684, pp. 68-87.  \\  % 47
S. Blondin et al. (2008), ApJ, 682, pp. 724-736.  \\  % 19
A. C. Becker et al. (2008), ApJL, 682, pp. 53-56. \\  % 6
T. Davis et al. (2007), ApJ, 666, pp. 716-725.  \\  % 476
W. M. Wood-Vasey et al. (2007), ApJ, 666, pp. 694-715. \\  % 746
G. Miknaitis et al. (2007), ApJ, 666, pp. 674-693. \\  % 224
S. Blondin et al. (2006), AJ, 131, pp. 1648-1666. \\  % 68
K. Krisciunas et al. (2005), AJ, 130, pp. 2453-2472. \\  % 42


%%%%%%%%%%%%%%%%%%%%%%%%%%%%%%%%%%%%%%%%%%%%%%%%%%%%%%%%%%%%%%%%%%%%%%

% OBSERVING RUN DETAILS - REQUIRED FOR EACH OBSERVING RUN REQUESTED
%
% For each run requested earlier in the \begin{obsruns}-\end{obsruns}
% sections of this proposal form, further run information must be
% specified.  Enter this block of information for each non-Gemini run.

% The \runid field must contain the run number plus 
% telescope/instrument-detector information as it appears for each 
% run in the obsruns sections. For example, 
% \runid{1}{KP-4m/ECHUV + T2KB}.

\runid{1}{WIYN/WHIRC}

% Describe the observations to be made during this observing run in
% the \technicaldescription section. Justify the specific telescope,
% the number of nights, the instrument, and the lunar phase requested.
% List objects, coordinates, and magnitudes (or surface brightness, if
% appropriate) using a LaTeX-coded table in this section or optionally
% enter the target information using the Target Tables described at
% the end of this template.  Target Tables of objects are required for 
% queue runs.
%
% Exposure time calculators (ETCs) for some optical and IR 
% instruments in use at CTIO and at KPNO are available to assist 
% you with the preparation of your proposal.  See the Web pages:
%    Imaging ETC      - https://na01.safelinks.protection.outlook.com/?url=http%3A%2F%2Fwww.noao.edu%2Fgateway%2Fccdtime%2F&data=01%7C01%7Ckap146%40pitt.edu%7C6f8fc2f751be47ae990208d3e317779d%7C9ef9f489e0a04eeb87cc3a526112fd0d%7C1&sdata=XOf3MYaxeOLMftiaGkmeLBUSn4NOSs70NzdslU6DjH8%3D&reserved=0
%    Spectroscopy ETC - https://na01.safelinks.protection.outlook.com/?url=http%3A%2F%2Fwww.noao.edu%2Fgateway%2Fspectime%2F&data=01%7C01%7Ckap146%40pitt.edu%7C6f8fc2f751be47ae990208d3e317779d%7C9ef9f489e0a04eeb87cc3a526112fd0d%7C1&sdata=N1EYgb9POJG2fBQr7tE64WTnu%2FtFLUtva4qJ%2BaWCZFI%3D&reserved=0

\technicaldescription

We need to obtain 2 host galaxy references. 
Observations need to be at least 3 times as long as the longest exposure of the field when the supernova was live (assuming 1" seeing). 
Our template observations need have better signal to noise than our supernova observations for accurate template subtractions. 
LSQ14xi was observed in J for 41 minutes:  its host galaxy needs 123 minutes which translates to five 5x5x15" scripts yielding 125 open shutter minutes.
ASASSN-15fs was observed in Ks for 16 minutes: its host galaxy needs 48 minutes which translates to two 5x5x15" scripts yielding 50 open shutter minutes.
Each dither script take $\sim$ 33 minutes to run for a total of 3.8 hours, plus $\sim40$ minutes to set up both fields (check focus, position, use WTTM if possible).
We need roughly an hour and a half for standard star observations.
This totals to approximately 6 hours or 1 night in April or May.
Any extra time can be filled with standard star observations.

Nights do not need to be photometric. 
The night sky is bright in the NIR regardless of the phase of the Moon; therefore, observations during bright time are acceptable. 

% Several instrument configuration parameters are requested.  Fill
% these in as appropriate for each run. 
%
% \begin{configuration}
% \filters{}            % List here any filters that you plan to use.
% \grating{}            % List any gratings/grisms need with this run.
% \order{}              % Specify any grating order(s).
% \crossdisperser{}     % List cross disperser, if needed.
% \slit{}               % Enter slit widths you plan to use.
% \multislit{}          % yes or no only
% \wstart{}             % Starting wavelength of wavelength range.
% \wend{}               % Ending wavelength of wavelength range.
% \cable{}              % For CTIO/Hydra: enter 2.0".  For WIYN: enter
%                         red, blue, or densepak.
% \corrector{}          % Enter red or blue for KP-coude, CAM5.
% \collimator{}         % Enter collimator needed.
% \adc{}                % If user selectable, enter yes or no only.
% \end{configuration}
%
% Details about these fields are available in the online help for the
% Web form at https://na01.safelinks.protection.outlook.com/?url=http%3A%2F%2Fwww.noao.edu%2Fnoaoprop%2Fhelp%2Fstandard.html%23iconfig&data=01%7C01%7Ckap146%40pitt.edu%7C6f8fc2f751be47ae990208d3e317779d%7C9ef9f489e0a04eeb87cc3a526112fd0d%7C1&sdata=Vc3HhVjcRHhtYDkbp9ZVvZLNaCZxs5AKZb7qRn1nn4M%3D&reserved=0

\begin{configuration}
\filters{J, H, Ks}
\grating{}
\order{}
\crossdisperser{}       
\slit{}
\multislit{}            
\wstart{}
\wend{}
\cable{}
\corrector{}            
\collimator{}             
\adc{}
\end{configuration}



% COORDINATE RANGES OF PRINCIPLE TARGETS   
% Use the \targetsra and \targetsdec fields to specify the range
% of right ascension (in hours) and declination (in degrees) of your
% principle targets for this observing run.
%
% For example:
%
%  \targetsra{14 to 17}
%  \targetsdec{-10 to 35}

\targetsra{12 to 19}			% RA range in hours
\targetsdec{-14 to +44}			% Dec. range in degrees



% Use \specialrequest to describe briefly any special or non-standard
% usage of instrumentation. 
\specialrequest


% If you plan to submit any Target Tables for this run they must be
% entered here.  Target Tables are required for all queue runs.

% TARGET TABLES
%
% Target Tables are required for all queue runs and are
% are optional for classically scheduled runs.  Target
% Tables, if included, are associated with each run in the observing
% run details section of the proposal form and must follow the 
% \specialrequest in each section.
%
% CT-1.3m tables require that all the fields in the target table be
% specified.  For other tables specific fields may be deleted EXCEPT 
% for the \obscomment command as mentioned below.
%
% Queue investigators should supply good coordinates
% with the \ra and \dec commands since these values may be used to
% observe your fields.
%
% Note that for iterative targets, only the parameters that need
% to be changed have to be specified.  Once a parameter is specified in
% a targettable environment, it is retained until explicitly changed.
%
% The \obscomment command is REQUIRED for each target entry and
% must be the last item; this command forces each target line to be
% printed.  If no comment is needed, leave the argument blank.
%
% The \begin{targettable}{} command for each table must contain the
% telescope/instrument-detector information for that particular run,
% i.e,. \begin{targettable}{KP-4m/ECHUV + T2KB}.
%
% HINTS: Long tables do not break across pages. If it is necessary to
% continue a table across a page you must start a new table.  Use
% /clearpage before the \begin{targettable} command for the new table.
%
\begin{targettable}{}
\objid{001}           % specify a 3-digit number for each target.
\object{LSQ14xi}          % 20 characters maximum
\ra{12:30:41.2}              % e.g., xx:xx:xx.x
\dec{-13:46:22}             % e.g., +-xx:xx:xx.x
\epoch{2000.00}           % e.g., 1950.3
\magnitude{unknown}
\filter{J}
\exptime{60}         % in seconds PER EXPOSURE
\nexposures{125}      % Number of exposures
\moondays{14}        % Days from new moon, use a number 0-14
\skycond{phot}         % "spec" or "phot"
\seeing{1}          % max allowable PSF FWHM (arcsecs)
\obscomment{}      % 20 characters maximum - REQUIRED COMMAND
\objid{002}           % specify a 3-digit number for each target.
\object{ASASSN-15fs}          % 20 characters maximum
\ra{18:58:40.8 }              % e.g., xx:xx:xx.x
\dec{+43:28:08}             % e.g., +-xx:xx:xx.x
\epoch{2000.00}           % e.g., 1950.3
\magnitude{12} % Observed in 2MASS
\filter{Ks}
\exptime{60}         % in seconds PER EXPOSURE
\nexposures{50}      % Number of exposures
\moondays{14}        % Days from new moon, use a number 0-14
\skycond{phot}         % "spec" or "phot"
\seeing{1}          % max allowable PSF FWHM (arcsecs)
\obscomment{}      % 20 characters maximum - REQUIRED COMMAND
\end{targettable}


%%%%%%%%%%%%%%%%%%%%%%%%%%%%%%%%%%%%%%%%%%%%%%%%%%%%%%%%%%%%%%%%%%

% Detailed information as noted below must be provided for all runs 
% up to SIX runs.  Since Gemini runs must be specified through the 
% Web form this section only applies to non-Gemini runs - use the 
% Web form to generate detailed information and target tables for 
% Gemini runs.  Target Tables are required for queue runs but
% are optional for classical observing.
%
%\runid{}{}
%\technicaldescription
%\begin{configuration}
%\filters{}
%\grating{}
%\order{}
%\crossdisperser{}
%\slit{}
%\multislit{}
%\wstart{}
%\wend{}
%\cable{}
%\corrector{}
%\collimator{}
%\adc{}
%\end{configuration}
%\specialrequest    % remove and not use for HET runs
%\begin{targettable}{}
%\objid{}           % specify a 3-digit number for each target
%\object{}          % 20 characters maximum
%\ra{}              % e.g., xx:xx:xx.x
%\dec{}             % e.g., +-xx:xx:xx.x
%\epoch{}           % e.g., 1950.3
%\magnitude{}
%\filter{}
%\exptime{}         % in seconds PER EXPOSURE
%\nexposures{}      % Number of exposures
%\moondays{}        % Days from new moon, use a number 0-14
%\skycond{}         % "spec" or "phot"
%\seeing{}          % max allowable PSF FWHM (arcsecs)
%\obscomment{}      % 20 characters maximum - REQUIRED COMMAND
%  - repeat target entry parameters as needed to complete Table -
%\end{targettable}

%%%%%%%%%%%%%%%%%%%%%%%%%%%%%%%%%%%%%%%%%%%%%%%%%%%%%%%%%%%%%%%%%%

% Please do not modify or delete this line.
\end{document}

% PROPOSAL SUBMISSION (if not submitting from the online proposal form):
%
%
% Upload this completed proposal latex file and figures at: 
%            https://na01.safelinks.protection.outlook.com/?url=http%3A%2F%2Fwww.noao.edu%2Fnoaoprop%2Fsubmit%2F&data=01%7C01%7Ckap146%40pitt.edu%7C6f8fc2f751be47ae990208d3e317779d%7C9ef9f489e0a04eeb87cc3a526112fd0d%7C1&sdata=MLKb%2FXAabunU%2FZANzvkzworL8Bcb6emrdrP1bHi72sU%3D&reserved=0
%
%
% Thank you for your interest in NOAO.  Contact us at
% noaoprop-help@noao.edu if you have any suggestions or comments.
